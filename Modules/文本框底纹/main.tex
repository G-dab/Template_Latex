\documentclass[utf-8]{article}
\usepackage{newtxtext}
\usepackage[dvipsnames,svgnames,x11names]{xcolor} %颜色宏包
\usepackage{ctex} %中文宏包
\usepackage{geometry} %页面布局
\usepackage{lipsum} %生成文字
\usepackage[strict]{changepage} % 提供一个 adjustwidth 环境
\usepackage{framed} % 实现方框效果
% ------------------** 页面布局 **------------------
\geometry{a4paper,centering,scale=0.8}
% environment derived from framed.sty: see leftbar environment definition

% ------------------*** 文本框设置 **-------------------
% 颜色定义
\definecolor{formalshade}{rgb}{0.95,0.95,1}
\definecolor{greenshade}{rgb}{0.90,0.99,0.91} %绿色文本框,竖线颜色设为 Green
\definecolor{redshade}{rgb}{1.00,0.90,0.90} %红色文本框,竖线颜色设为 LightCoral
\definecolor{brownshade}{rgb}{0.99,0.97,0.93} %莫兰迪棕色,竖线颜色设为 BurlyWood
% 注意行末需要把空格注释掉,不然画出来的方框会有空白竖线
%蓝紫色
\newenvironment{Purpoe}{%
	\def\FrameCommand{%
		\hspace{1pt}%
		{\color{DarkBlue}\vrule width 2pt}%
		{\color{formalshade}\vrule width 4pt}%
		\colorbox{formalshade}%
	}%
	\MakeFramed{\advance\hsize-\width\FrameRestore}%
	\noindent\hspace{-4.55pt}% disable indenting first paragraph
	\begin{adjustwidth}{}{7pt}%
		\vspace{2pt}\vspace{2pt}%
	}
	{%
		\vspace{2pt}\end{adjustwidth}\endMakeFramed%
}
%棕色
\newenvironment{Brown}{%
	\def\FrameCommand{%
		\hspace{1pt}%
		{\color{BurlyWood}\vrule width 2pt}%
		{\color{brownshade}\vrule width 4pt}%
		\colorbox{brownshade}%
	}%
	\MakeFramed{\advance\hsize-\width\FrameRestore}%
	\noindent\hspace{-4.55pt}% disable indenting first paragraph
	\begin{adjustwidth}{}{7pt}%
		\vspace{2pt}\vspace{2pt}%
	}
	{%
		\vspace{2pt}\end{adjustwidth}\endMakeFramed%
}
% ------------------** 正文 ***-------------------
\begin{document}

\section{颜色测试}
    \subsection{Base Color}
        {\color{black} black}
        {\color{darkgray} 灰色}
        {\color{lime} 清柠檬}
        {\color{red} 红色}
        {\color{purple} 紫色}
        {\color{blue} 蓝色}
    \subsection{dvipsnames}
        {\color{Apricot} 杏黄色}
        {\color{Cyan} 青色}
    \subsection{svgnames}
        {\color{AliceBlue} AliceBlue}
        {\color{DarkKhaki} DarkKhaki}
    \subsection{x11names}
        {\color{AntiqueWhite1} AntiqueWhite1}
        {\color{AntiqueWhite2} AntiqueWhite2}

\section{文本框测试}
\subsection{注释类文本框}
	\begin{Brown}
		\textbf{中文注释}
		\lipsum[4]
	\end{Brown}
	\begin{Purpoe}
		\textbf{English Note}
		\lipsum[5]
	\end{Purpoe}
\subsection{定理类文本框} %这里用tcolorbox宏包
	% 定义
	\begin{tcolorbox}
		[colback=Emerald!10,colframe=cyan!40!black,title=\textbf{Definition}]
	\end{tcolorbox}
	% 证明
	\begin{tcolorbox}[colback=JungleGreen!10!Cerulean!15,colframe=CornflowerBlue!60!Black]
		\lipsum[3]
	\end{tcolorbox}
	% 定理
	\begin{tcolorbox}[title=\textbf{Linear Dependence Lemma},colback=SeaGreen!10!CornflowerBlue!10,colframe=RoyalPurple!55!Aquamarine!100!]
		\lipsum[3]
	\end{tcolorbox}
	% 性质
	\begin{tcolorbox}[title = \textbf{Propertiea}, colback=Salmon!20, colframe=Salmon!90!Black]
		\lipsum[3]
	\end{tcolorbox}
	% 超级重要
	\begin{tcolorbox}[title=\textbf{Warning}, colback=red!5,colframe=red!75!black]
	\end{tcolorbox}
\end{document}