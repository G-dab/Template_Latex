\subsection{相似}
\begin{tcolorbox}
    [colback=Emerald!10,colframe=cyan!40!black,title=\textbf{相似}]
    若$P^{-1}AP=B$,其中$P$为可逆阵,则称$A$与$B$相似    
\end{tcolorbox}
\begin{tcolorbox}[title=\textbf{矩阵相似的性质},colback=SeaGreen!10!CornflowerBlue!10,colframe=RoyalPurple!55!Aquamarine!100!]
    若$P^{-1}AP=B$,则对任意多项式$f(x)$,有$P^{-1}f(A)P=f(B)$
\end{tcolorbox}
\begin{Chinese Note}
    \textbf{为什么要研究相似?} 相似可以将矩阵分解为对角矩阵和可逆矩阵的乘积,从而简化矩阵的运算
\end{Chinese Note}

\begin{tcolorbox}
    [colback=Emerald!10,colframe=cyan!40!black,title=\textbf{特征值和特征向量}]
    \textbf{特征值} $\lambda$ : $Ax=\lambda x$的解,其中$x$为非零向量\\
    \textbf{特征向量} $x$ : $Ax=\lambda x$的解,其中$\lambda$为特征值\\
    \textbf{特征多项式} : $f(\lambda)=|A-\lambda I|$\\
    \textbf{特征方程} : $f(\lambda)=0$\\
    \textbf{代数重数} : 特征值$\lambda$在特征多项式$f(\lambda)$中的重数\\
    \textbf{几何重数} : 特征值$\lambda$对应的特征向量的个数(特征子空间的维数)
\end{tcolorbox}
\begin{itemize}
    \item A相似于对角阵(可对角化) $\iff$ A有n个线性无关的特征向量
    \item $\sum_{i=1}^{n} \lambda_i = tr(A)$ \qquad $\prod_{i=1}^{n} \lambda_i = |A|$
    \item 相似矩阵有相同的特征多项式、特征值、行列式、迹
    \item 对任意特征值$\lambda$,其几何重数$\leqslant$代数重数 ($\star$)
    \item A相似于对角阵(可对角化) $\iff$ 每个特征值的代数重数等于几何重数
\end{itemize}

\begin{Chinese Note}
    \textbf{可对角化矩阵的对角化方法}
    \begin{enumerate}
        \item 解特征方程$f(\lambda)=0$得到特征值$\lambda_1,\lambda_2,\cdots,\lambda_n$,对应代数重数为$n_1,n_2,\cdots,n_s$
        \item 对每一个$\lambda_i$, 求$(\lambda_i I-A)x=0$的基础解系,得到线性无关的特征向量$\alpha_{i1},\alpha_{i2},\cdots,\alpha_{in_i}$ 
        \item 取$P=(\alpha_{11},\alpha_{12},\cdots,\alpha_{1n_1},\alpha_{21},\alpha_{22},\cdots,\alpha_{2n_2},\cdots,\alpha_{s1},\alpha_{s2},\cdots,\alpha_{sn_s})$,
        则$P^{-1}AP=\Lambda=
        \left(\begin{array}{ccc}   
                    \lambda_1E_{n_1} &  &\\
                     & \cdots& \\
                     &  &  \lambda_sE_{n_s}\\
                \end{array}\right)$                 
    \end{enumerate}
\end{Chinese Note}

\subsection{特征值估计}
\begin{tcolorbox}
    [colback=Emerald!10,colframe=cyan!40!black,title=\textbf{Gerschgorin圆盘}]
    设$A=(a_{ij})$为n阶复矩阵,\\
    \textbf{(Gerschgorin行区域)} $D_{i}(A) = {z||z-a_{ij}|\leq \sum_{i\neq j}a_{ij}}$\\
    列区域的定义也类似
\end{tcolorbox}

\begin{tcolorbox}[title=\textbf{圆盘定理},colback=SeaGreen!10!CornflowerBlue!10,colframe=RoyalPurple!55!Aquamarine!100!]
    \textbf{(第一圆盘定理)} 矩阵A的所有特征值都落在Gerschgorin圆盘的并集(Gerschgorin区域)中(是Gerschgorin行区域和列区域的交集)
    ($\star$)\\
    \textbf{(第二圆盘定理)} 若C是Gerchgorin区域中由K个圆盘组成的连通分支,则C内恰有K个特征值
    ($\star$)
\end{tcolorbox}
\textbf{分离圆盘法} : 若$D_i(A)\cap D_j(A)=\varnothing$,则$D_i(A)$和$D_j(A)$内各有一个特征值

