\subsection{初等积分法}
($\star$)
\subsection{线性方程}
\begin{enumerate}
    \item 存在性与唯一性
    \begin{enumerate}
        \item ($\star$)假设$A(t)$是区间$[\alpha, \beta]$上的连续矩阵函数, $f(t)$是区间$[\alpha, \beta]$上的连续列向量函数, 
        则初值问题$\frac{d}{dt}x=Ax+f(t), x(t_0)=x^0$在区间$[\alpha, \beta]$上存在唯一解
        \item 齐次方程
    \end{enumerate}
\end{enumerate}
\subsection{常系数线性方程}
线性微分方程求解的关键就在通解的求得, 而常系数下通解有办法求得
\begin{enumerate}
    \item 齐次通解 -- Euler待定指数法\\
    证明这些解线性无关($\star$)
    \begin{enumerate}
        \item 列出特征方程$P(\lambda)=0$,求出特征根
        \item (1) $P(\lambda)=0$有n个互异实根
        \item (2) $P(\lambda)=0$有r个互异实根
        \item (3) $P(\lambda)=0$有r个互异实根和l对共轭互异复根
    \end{enumerate}
    \item 非齐次特解 -- 算子法\\
    算子解法比常数变易公式更简便些
    \begin{enumerate}
        \item 解析展开法
        \item 代换法
        \item 二项式法
    \end{enumerate}
    \item 常系数线性方程组
    根据矩阵$A$的特征值得到方程的通解
    \begin{enumerate}
        \item ($\star$) 若A有n个互异实特征根, 则方程组$\frac{d}{dt}x=Ax$有基本解组$e^{\lambda_{1}t}c_1, e^{\lambda_{2}t}c_2, \cdots, e^{\lambda_{n}t}c_n$,其中$c_i$为$\lambda_i$的特征向量
        \item 矩阵指数函数$e^{tA}$ (一定收敛)
        \item ($\star$) 齐次方程组$\frac{d}{dt}x =Ax$有标准解矩阵$e^{tA}$, 
        非齐次方程$\frac{d}{dt}x=Ax+f(t)$的通解为$x=e^{tA}c + \int_{t_0}^{t} e^{(t-s)A}f(s)ds$
        \item 标准解矩阵的初等表达: 把$e^{tA}$转换成有限和的形式算出来
    \end{enumerate}
    \item 应用:机械振动
    \begin{enumerate}
        \item 无阻尼自由振动 $\ddot{x}+\omega^2 x=0$
        \item 有阻尼自由振动 $\ddot{x}+2\delta \dot{x}+\omega^2 x=0$
        \item 无阻尼强迫振动 $\ddot{x}+\omega^2 x = \frac{F_0}{m}\cos pt$
        \item 有阻尼强迫振动 $\ddot{x}+2\delta \dot{x}+\omega^2 x = \frac{F_0}{m}\cos pt$
    \end{enumerate}
\end{enumerate}

\subsection{一般理论}
\begin{enumerate}
    \item Picard存在唯一性定理
    \begin{itemize}
        \item $\dot{x}=f(t,x), x(t_0)=x_0$,其中$f(t,x)$连续且满足Lipschiz条件,
        则初值问题在区间$|t-t_0|\leq h$存在唯一解
        \item 构造Picard序列 -- 求近似解 $\varphi_{n}(t) = x_0 + \int_{t_0}^{t}f(\tau , \varphi_{n-1}(\tau))d\tau$
        \item Picard序列误差估计 $|\varphi_{n}(t) - \varphi(t)| \leq \frac{ML^{n}}{(n+1)!}h^{n+1}$
    \end{itemize}
    \item Peano存在性定理
    \begin{itemize}
        \item \textbf{Ascoli-Arzela定理} $[\alpha, \beta]$上的一致有界且等度连续的函数列${f_k(t)}$有一致收敛的子序列
        \item $\dot{x}=f(t,x), x(t_0)=x_0$,其中$f(t,x)$连续,
        则初值问题在区间$|t-t_0|\leq h$有至少一个解 (没有Lipschiz条件,解仍存在但是不唯一)
        \item 构造Eule折线, 证明其是$\epsilon$-逼近解
    \end{itemize}
    \item 解的延拓
    \begin{itemize}
        \item 饱和解的存在区间为开区间
        \item \textbf{解的延拓定理} G为$\mathbb{R}^2$上的开区间, $f(t,x)$在G内连续,
        则对$\dot{x}=f(t,x)$的任一饱和解$\varphi(t)$必能到达G的边界
        \item 找一个比函数连续区域更小的区域,得到矛盾
    \end{itemize}
    \item 微分不等式与比较定理
    \begin{itemize}
        \item ($\star$) \textbf{Gronwall不等式} x(t), f(t)在$[t_0, t_1]$上连续,$f(t)\geq 0$, 若$x(t)\leq g + \int_{t_0}^{t}f(s)x(s)ds$, 则$x(t)\leq g\exp(\int_{t_0}^{t}f(s)ds)$
        \item \textbf{推广Gronwall不等式} x(t), f(t), g(t)在$[t_0, t_1]$上连续,$f(t)\geq 0$, 若$x(t)\leq g(t) + \int_{t_0}^{t}f(s)x(s)ds$, 则$x(t)\leq g(t) + \int_{t_0}^{t}f(\tau)g(\tau)e^{\int_{\tau}^{t}f(s)ds}d\tau$
        \item ($\star$) \textbf{第一比较定理}, 证明时注意描述
        \item ($\star$) 最大解和最小解的存在性和唯一性, 可延拓性, 证明时构造递减序列
        \item ($\star$) \textbf{第二比较定理}, 证明时构造递减序列
    \end{itemize}
    \item 解对初值和参数的依赖
    \item 微分方程数值解
\end{enumerate}
\subsection{定性理论初步}
\begin{enumerate}
    \item 动力系统概念
    \item Lyapunov稳定性
    \item Lyapunov直接法
    \item 平面平衡点分析
    \item 周期轨道与Poincare映射
    \item 平面Hamilton系统
\end{enumerate}
