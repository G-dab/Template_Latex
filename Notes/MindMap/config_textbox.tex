% ---
% Module Name: textbox.tex
% description: 文本框底纹配置文件
% ---
\usepackage{tcolorbox}
% 颜色定义
\definecolor{formalshade}{rgb}{0.95,0.95,1}
\definecolor{greenshade}{rgb}{0.90,0.99,0.91} %绿色文本框,竖线颜色设为 Green
\definecolor{redshade}{rgb}{1.00,0.90,0.90} %红色文本框,竖线颜色设为 LightCoral
\definecolor{brownshade}{rgb}{0.99,0.97,0.93} %莫兰迪棕色,竖线颜色设为 BurlyWood
% 注意行末需要把空格注释掉,不然画出来的方框会有空白竖线
%蓝紫色
\newenvironment{Chinese Note}{%
	\def\FrameCommand{%
		\hspace{1pt}%
		{\color{DarkBlue}\vrule width 2pt}%
		{\color{formalshade}\vrule width 4pt}%
		\colorbox{formalshade}%
	}%
	\MakeFramed{\advance\hsize-\width\FrameRestore}%
	\noindent\hspace{-4.55pt}% disable indenting first paragraph
	\begin{adjustwidth}{}{7pt}%
		\vspace{2pt}\vspace{2pt}%
	}
	{%
		\vspace{2pt}\end{adjustwidth}\endMakeFramed%
}
%棕色
\newenvironment{English Note}{%
	\def\FrameCommand{%
		\hspace{1pt}%
		{\color{BurlyWood}\vrule width 2pt}%
		{\color{brownshade}\vrule width 4pt}%
		\colorbox{brownshade}%
	}%
	\MakeFramed{\advance\hsize-\width\FrameRestore}%
	\noindent\hspace{-4.55pt}% disable indenting first paragraph
	\begin{adjustwidth}{}{7pt}%
		\vspace{2pt}\vspace{2pt}%
	}
	{%
		\vspace{2pt}\end{adjustwidth}\endMakeFramed%
}
